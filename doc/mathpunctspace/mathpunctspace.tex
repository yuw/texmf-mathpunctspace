\documentclass{article}

\usepackage{mathpunctspace}

\title{Control the Space After Punctuation in Expressions}
\author{Yuwsuke Kieda}
\date{2017/04/03 v0.4}

\begin{document}
\maketitle

\section{Descriptions}

We provide a mechanism to control the space after the comma in the expressions.

\section{Version}

0.4

\section{Usage}

\subsection{Sample of Preamble}

\noindent
\verb!\!\texttt{usepackage[comma=natural,semicolon=natural]}\verb!{mathpunctspace}!

\noindent
\verb!\!\texttt{usepackage[comma=$N$mu,semicolon=$M$mu]}\verb!{mathpunctspace}!

\noindent
\verb!\!\texttt{usepackage[unit=pt,comma=$N$pt,semicolon=$M$pt]}\verb!{mathpunctspace}!

\subsection{Options}

\begin{itemize}
 \item unit: mu or other (default = mu)
 \item comma: substitute keyword natural or skip (default = natural)
 \item semicolon: substitute keyword natural or skip (default = natural)
 \item latexorg: original behavior of LaTeX
\end{itemize}

Remark: keyword ``natural'' mean the spacing of the in-line.

\section{License}

BSD 2-Clause License

\section{Repository}

\texttt{https://github.com/yuw/texmf-mathpunctspace}

\section{Sample}

\begin{verbatim}
\usepackage[comma=natural,semicolon=natural]{mathpunctspace}
\end{verbatim}

Lorem ipsum dolor sit amet $(x, y)$, consectetuer adipiscing elit.

Lorem ipsum dolor sit amet $\{x; x \in A\}$; consectetuer adipiscing elit.

\end{document}